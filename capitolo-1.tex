\chapter{Machine Learning}
\nocite{Nilsson96introductionto}Si intende, per ML, una tecnica utilizzabile per risolvere problemi  nei quali si è in grado di specificare degli output dati determinati input senza però essere in grado di comprendere la relazione esistente tra i valori. Spesso problemi del tipo appena descritto portano con se un quantitativo di dati decisamente troppo vasto per essere analizzato soltanto da una o più persone, questa particolare tecnica offre dunque un modo più efficiente per svolgere analisi sui dati e rendere automatici determinati procedimenti consentendo anche di diminuire errori di tipo casuale dati dall'uomo, o addirittura catturare più informazioni rispetto a un operatore umano.\\
Il ML possiede un dominio di applicazione molto ampio, che può andare dalla medicina alla psicologia, e proprio per questo motivo esistono vari approcci e modalità. Nel mio lavoro di tirocinio in particolare è stato usato un approccio supervisionato. 
\section{Apprendimento supervisionato}
Lo scopo di questo sistema è come dice la parola stessa, di supervisionare una macchina, questo costruendo un \emph{data-set}  \footnote{Il data-set è un insieme di dati raccolti da misurazioni svolte in fase di preparazione per lo studio del problema, classificati in modo opportuno.}
di valori, solitamente dei tipi che seguono: numerici, che devono essere normalizzati \footnote{Per normalizzazione si intende un processo che rende paragonabili dati tra loro di diversa natura fornendo valori compresi tra 0 e 1, ciò ottenuto nel modo seguente:\newline $\forall x_{i}, i\subset\mathbb{N}$ $x_{norm} = \dfrac{x_{i}-x_{min}}{x_{max} - x_{min}} $} prima di essere usati, nominali, stringhe oppure date, che serviranno come punto di riferimento alla macchina per costruire le regole che permetteranno di restituire precisi output dati determinati input.\\
Il ML con apprendimento supervisionato si divide in problemi di classificazione o di regressione. Il Caso in questione rientra nel tipo di classificazione, dunque un problema che dato un valore, o insieme di valori, restituisce un risultato discreto appoggiandosi a un albero di decisione che la macchina ha costruito a partire da un meta-algoritmo, ciò dopo essere stata allenata con i dati di interesse.
\section{Alberi di decisione} 
Un albero di decisione è una tecnica di ML supervisionato, i cui nodi interni sono dei test sui valori forniti, e el foglie sono le categorie a cui possono appartenere i valori di input \cite{Nilsson96introductionto}.
Questa struttura decisionale viene costruita tramite dei \emph{meta-algoritmi}\footnote{Ovvero si tratta di un algoritmo che non risolve il problema, ma è in grado di costruire lui stesso un nuovo algoritmo.} ad esempio J48, usato nel nostro caso, un'estensione del suo predecessore ID3 che prevede i seguenti passaggi:
\begin{enumerate}
	\item Iterazione di ogni attributo avviene il calcolo dell'\emph{entropia}, o \emph{inforamtion gain}, valori compresi tra 0 e 1.
	\item Selezione dell'attributo con l'entropia minore, o information gain maggiore.
	\item Divisione del data-set sull'attributo selezionato, generando due sub-set.
\end{enumerate}
Questi passaggi teoricamente andrebbero ripetuti per ogni sub-set fino a ottenerne di puri\footnote{Dove ogni uno di essi contiene una sola sola tupla}, operazione però non sempre possibile. Vengono dunque presi in esame vari fattori come la dimensione del data-set, o il livello di precisione accettabile per interrompere l'esecuzione dell'algoritmo. Un altro caso possibile di interruzione si ha nel momento in cui viene calcolata un'entropia pari a 1.

\section{I passaggi necessari}
Per costruire un sistema di classificazione intelligente con l'utilizzo del ML supervisionato, precisamente con weka sono dunque necessari vari passaggi. Il primo passaggio è la creazione di un data-set, questo potrà essere fatto in diversi modi, nel caso corrente il risultato è un file \href{https://www.cs.waikato.ac.nz/ml/weka/arff.html}{arff}\footnote{File di tipo testuale che presenta due sezioni, header e data. In header viene specificata la struttura e il tipo di dati per ogni tupla. In data vengono riportate le misurazioni o valori ordinati come specificato in header con i rispettivi valori.} contenente le features\footnote{Sono le caratteristiche di interesse del fenomeno che deve essere studiato} dei file audio che sono stati etichettati con le rispettive classi.\\
Segue poi una fase di allenamento dove viene costruito l'albero di decisione con un algoritmo, utilizzabile poi come strumento per predire dato un input sconosciuto, quale sarà la sua classe. Questi ultimi due passaggi saranno eseguiti utilizzando lo stesso meta-algoritmo.`
%, in questo particolare caso è stato usato per classificare inizialmente \emph{nota} o \emph{non-nota}, dati dei campioni audio. Questo  evoluto per restituire 5 classi che indicano differenti gradi di essere nota o meno\footnote{La classe 5 significa essere certamente una nota, la classe 1 rappresenta invece una non nota, tutti i valori di mezzo rappresentano i diversi gradi tra l'essere nota o meno}.
