\chapter{Machine Learning}
Si intende, per ML, una tecnica utilizzabile per risolvere problemi  nei quali si è in grado di specificare degli output dati determinati input senza però essere in grado di comprendere la relazione esistente tra i valori. Spesso problemi del tipo appena descritto portano con se un quantitativo di dati decisamente troppo vasto per essere analizzato soltanto da una o più persone, questa particolare tecnica offre dunque un modo più efficiente per svolgere analisi sui dati e rendere automatici determinati procedimenti consentendo anche di diminuire errori di tipo casuale dati dall'uomo, o addirittura catturare più informazioni rispetto a un operatore.\\
Im ML possiede un dominio di applicazione molto ampio, che può andare dalla medicina alla psicologia, e proprio per questo motivo esistono vari approcci e modalità. Nel mio lavoro di tirocinio in particolare è stato usato un approccio supervisionato. 
\section{Apprendimento supervisionato}
Lo scopo di questo sistema è come dice la parola stessa, di supervisionare una macchina, questo costruendo un data-set \footnote{Il data-set è un insieme di dati raccolti da misurazioni svolte in fase di preparazione per lo studio del problema, classificati in modo opportuno.}
di valori solitamente dei tipi che seguono: numerici (che devono essere normalizzati \footnote{Per normalizzazione si intende un processo che rende paragonabili dati tra loro di diversa natura fornendo valori compresi tra 0 e 1, ciò ottenuto nel modo seguente:\newline $\forall x_{i}, i\subset\mathbb{N}$ $x_{norm} = \dfrac{x_{i}-x_{min}}{x_{max} - x_{min}} $} prima di essere usati), nominali, stringhe oppure date, che serviranno come punto di riferimento alla macchina per costruire le regole che permetteranno di restituire precisi output dati determinati input.\\
Il ML con apprendimento supervisionato si divide in problemi di classificazione o di regressione. Il Caso in questione rientra nel tipo di classificazione, dunque un problema che dato un valore, o insieme di valori, restituisce un risultato discreto appoggiandosi a un albero di decisione che la macchina ha costruito a partire da un meta-algoritmo, ciò dopo essere stata allenata con i dati di interesse.
\section{Regole ed alberi} 
Un albero di decisione è un insieme DEFINIZIONE A PAGINA 82 DI 188 DI NILSSON BOOK
parla di J48

SITI\\
https://www.cs.waikato.ac.nz/ml/weka/arff.html

LIBRI ARTICOLI\\
https://ai.stanford.edu/~nilsson/MLBOOK.pdf

