\chapter{L'intelligenza}
La fase di costruzione dell'intelligenza in grado di comprendere quando vengono suonate le note in una traccia audio si è divisa in due fasi. La prima è stata quella di riconoscere la differenza tra nota e non nota dati dei audio selezionati manualmente senza considerare la lunghezza. In un secondo momento si è poi proceduto dividendo i campioni di note in più parti di uguale lunghezza, associando a ogni divisione del campione un valore che va da 1 a 5, ovvero delle classi che indicano, essere certamente una nota associandovi la classe 5, o essere certamente non una nota accoppiando il campione alla classe 1.\\
Le tracce di strumenti fornitemi dal tutore e utilizzate per gli esperimenti sono state quelle di: 
\begin{itemize}
	\item Contrabbasso
	\item Spazzolata di rullante
	\item Colpo di rullante
	\item Colpo di cassa
\end{itemize}


\section{Prima fase}
In questa fase lo scopo è stato appunto comprendere se una macchina con le features selezionate fosse in grado di riconoscere la differenza tra nota e non nota. Una volta creato il file arff contenente le informazioni necessarie tramite un apposito script python lo stesso file è stato usato per costruire l'albero decisionale utilizzando l'algoritmo J48 (versione più fine di ID3 descritto nel Capitolo 2) ottenendo come risultato le seguenti matrici di confusione per gli strumenti  sopra elencati :\\


%TODO: RICONTROLLA MATRICI
% MATRICE DI CONFSIONE DEL CONTRABBASSO
\def\myConfMat{{
		{ 114,  10},  %row 1
		{   7,  84},  %row 2
}}

\def\classNames{{"SI","NO"}} %class names. Adapt at will

\def\numClasses{2} %number of classes. Could be automatic, but you can change it for tests.

\def\myScale{1.5} % 1.5 is a good scale. Values under 1 may need smaller fonts!
\begin{tikzpicture}[
scale = \myScale,
%font={\scriptsize}, %for smaller scales, even \tiny may be useful
]

\tikzset{vertical label/.style={rotate=90,anchor=east}}   % usable styles for below
\tikzset{diagonal label/.style={rotate=45,anchor=north east}}

\foreach \y in {1,...,\numClasses} %loop vertical starting on top
{
	% Add class name on the left
	\node [anchor=east] at (0.4,-\y) {\pgfmathparse{\classNames[\y-1]}\pgfmathresult}; 
	
	\foreach \x in {1,...,\numClasses}  %loop horizontal starting on left
	{
		%---- Start of automatic calculation of totSamples for the column ------------   
		\def\totSamples{0}
		\foreach \ll in {1,...,\numClasses}
		{
			\pgfmathparse{\myConfMat[\ll-1][\x-1]}   %fetch next element
			\xdef\totSamples{\totSamples+\pgfmathresult} %accumulate it with previous sum
			%must use \xdef fro global effect otherwise lost in foreach loop!
		}
		\pgfmathparse{\totSamples} \xdef\totSamples{\pgfmathresult}  % put the final sum in variable
		%---- End of automatic calculation of totSamples ----------------
		
		\begin{scope}[shift={(\x,-\y)}]
		\def\mVal{\myConfMat[\y-1][\x-1]} % The value at index y,x (-1 because of zero indexing)
		\pgfmathtruncatemacro{\r}{\mVal}   %
		\pgfmathtruncatemacro{\p}{round(\r/\totSamples*100)}
		\coordinate (C) at (0,0);
		\ifthenelse{\p<50}{\def\txtcol{black}}{\def\txtcol{white}} %decide text color for contrast
		\node[
		draw,                 %draw lines
		text=\txtcol,         %text color (automatic for better contrast)
		align=center,         %align text inside cells (also for wrapping)
		fill=black!\p,        %intensity of fill (can change base color)
		minimum size=\myScale*10mm,    %cell size to fit the scale and integer dimensions (in cm)
		inner sep=0,          %remove all inner gaps to save space in small scales
		] (C) {\r\\\p\%};     %text to put in cell (adapt at will)
		%Now if last vertical class add its label at the bottom
		\ifthenelse{\y=\numClasses}{
			\node [] at ($(C)-(0,0.75)$) % can use vertical or diagonal label as option
			{\pgfmathparse{\classNames[\x-1]}\pgfmathresult};}{}
		\end{scope}
	}
}
%Now add x and y labels on suitable coordinates
\coordinate (yaxis) at (-0.3,0.5-\numClasses/2);  %must adapt if class labels are wider!
\coordinate (xaxis) at (0.5+\numClasses/2, -\numClasses-1.25); %id. for non horizontal labels!
\node [vertical label] at (yaxis) {Classe Risultante};
\node []               at (xaxis) {Classe Target};
\end{tikzpicture}


%% MATRICDE DI CONFUSIONE DELLA CASSA
\def\myConfMat{{
		{ 132,   1},  %row 1
		{   0,  71},  %row 2
}}

\def\classNames{{"SI","NO"}} %class names. Adapt at will

\def\numClasses{2} %number of classes. Could be automatic, but you can change it for tests.

\def\myScale{1.5} % 1.5 is a good scale. Values under 1 may need smaller fonts!
\begin{tikzpicture}[
scale = \myScale,
%font={\scriptsize}, %for smaller scales, even \tiny may be useful
]

\tikzset{vertical label/.style={rotate=90,anchor=east}}   % usable styles for below
\tikzset{diagonal label/.style={rotate=45,anchor=north east}}

\foreach \y in {1,...,\numClasses} %loop vertical starting on top
{
	% Add class name on the left
	\node [anchor=east] at (0.4,-\y) {\pgfmathparse{\classNames[\y-1]}\pgfmathresult}; 
	
	\foreach \x in {1,...,\numClasses}  %loop horizontal starting on left
	{
		%---- Start of automatic calculation of totSamples for the column ------------   
		\def\totSamples{0}
		\foreach \ll in {1,...,\numClasses}
		{
			\pgfmathparse{\myConfMat[\ll-1][\x-1]}   %fetch next element
			\xdef\totSamples{\totSamples+\pgfmathresult} %accumulate it with previous sum
			%must use \xdef fro global effect otherwise lost in foreach loop!
		}
		\pgfmathparse{\totSamples} \xdef\totSamples{\pgfmathresult}  % put the final sum in variable
		%---- End of automatic calculation of totSamples ----------------
		
		\begin{scope}[shift={(\x,-\y)}]
		\def\mVal{\myConfMat[\y-1][\x-1]} % The value at index y,x (-1 because of zero indexing)
		\pgfmathtruncatemacro{\r}{\mVal}   %
		\pgfmathtruncatemacro{\p}{round(\r/\totSamples*100)}
		\coordinate (C) at (0,0);
		\ifthenelse{\p<50}{\def\txtcol{black}}{\def\txtcol{white}} %decide text color for contrast
		\node[
		draw,                 %draw lines
		text=\txtcol,         %text color (automatic for better contrast)
		align=center,         %align text inside cells (also for wrapping)
		fill=black!\p,        %intensity of fill (can change base color)
		minimum size=\myScale*10mm,    %cell size to fit the scale and integer dimensions (in cm)
		inner sep=0,          %remove all inner gaps to save space in small scales
		] (C) {\r\\\p\%};     %text to put in cell (adapt at will)
		%Now if last vertical class add its label at the bottom
		\ifthenelse{\y=\numClasses}{
			\node [] at ($(C)-(0,0.75)$) % can use vertical or diagonal label as option
			{\pgfmathparse{\classNames[\x-1]}\pgfmathresult};}{}
		\end{scope}
	}
}
%Now add x and y labels on suitable coordinates
\coordinate (yaxis) at (-0.3,0.5-\numClasses/2);  %must adapt if class labels are wider!
\coordinate (xaxis) at (0.5+\numClasses/2, -\numClasses-1.25); %id. for non horizontal labels!
\node [vertical label] at (yaxis) {Classe Risultante};
\node []               at (xaxis) {Classe Target};
\end{tikzpicture}

%% MATRICDE DI CONFUSIONE DEL RULLANTE SPAZZOLATO
\def\myConfMat{{
		{  67,  10},  %row 1
		{  13,  63},  %row 2
}}

\def\classNames{{"SI","NO"}} %class names. Adapt at will

\def\numClasses{2} %number of classes. Could be automatic, but you can change it for tests.

\def\myScale{1.5} % 1.5 is a good scale. Values under 1 may need smaller fonts!
\begin{tikzpicture}[
scale = \myScale,
%font={\scriptsize}, %for smaller scales, even \tiny may be useful
]

\tikzset{vertical label/.style={rotate=90,anchor=east}}   % usable styles for below
\tikzset{diagonal label/.style={rotate=45,anchor=north east}}

\foreach \y in {1,...,\numClasses} %loop vertical starting on top
{
	% Add class name on the left
	\node [anchor=east] at (0.4,-\y) {\pgfmathparse{\classNames[\y-1]}\pgfmathresult}; 
	
	\foreach \x in {1,...,\numClasses}  %loop horizontal starting on left
	{
		%---- Start of automatic calculation of totSamples for the column ------------   
		\def\totSamples{0}
		\foreach \ll in {1,...,\numClasses}
		{
			\pgfmathparse{\myConfMat[\ll-1][\x-1]}   %fetch next element
			\xdef\totSamples{\totSamples+\pgfmathresult} %accumulate it with previous sum
			%must use \xdef fro global effect otherwise lost in foreach loop!
		}
		\pgfmathparse{\totSamples} \xdef\totSamples{\pgfmathresult}  % put the final sum in variable
		%---- End of automatic calculation of totSamples ----------------
		
		\begin{scope}[shift={(\x,-\y)}]
		\def\mVal{\myConfMat[\y-1][\x-1]} % The value at index y,x (-1 because of zero indexing)
		\pgfmathtruncatemacro{\r}{\mVal}   %
		\pgfmathtruncatemacro{\p}{round(\r/\totSamples*100)}
		\coordinate (C) at (0,0);
		\ifthenelse{\p<50}{\def\txtcol{black}}{\def\txtcol{white}} %decide text color for contrast
		\node[
		draw,                 %draw lines
		text=\txtcol,         %text color (automatic for better contrast)
		align=center,         %align text inside cells (also for wrapping)
		fill=black!\p,        %intensity of fill (can change base color)
		minimum size=\myScale*10mm,    %cell size to fit the scale and integer dimensions (in cm)
		inner sep=0,          %remove all inner gaps to save space in small scales
		] (C) {\r\\\p\%};     %text to put in cell (adapt at will)
		%Now if last vertical class add its label at the bottom
		\ifthenelse{\y=\numClasses}{
			\node [] at ($(C)-(0,0.75)$) % can use vertical or diagonal label as option
			{\pgfmathparse{\classNames[\x-1]}\pgfmathresult};}{}
		\end{scope}
	}
}
%Now add x and y labels on suitable coordinates
\coordinate (yaxis) at (-0.3,0.5-\numClasses/2);  %must adapt if class labels are wider!
\coordinate (xaxis) at (0.5+\numClasses/2, -\numClasses-1.25); %id. for non horizontal labels!
\node [vertical label] at (yaxis) {Classe Risultante};
\node []               at (xaxis) {Classe Target};
\end{tikzpicture}

%% MATRICDE DI CONFUSIONE DEL COLPO DI RULLANTE
\def\myConfMat{{
		{  63,   1},  %row 1
		{   1,  72},  %row 2
}}

\def\classNames{{"SI","NO"}} %class names. Adapt at will

\def\numClasses{2} %number of classes. Could be automatic, but you can change it for tests.

\def\myScale{1.5} % 1.5 is a good scale. Values under 1 may need smaller fonts!
\begin{tikzpicture}[
scale = \myScale,
%font={\scriptsize}, %for smaller scales, even \tiny may be useful
]

\tikzset{vertical label/.style={rotate=90,anchor=east}}   % usable styles for below
\tikzset{diagonal label/.style={rotate=45,anchor=north east}}

\foreach \y in {1,...,\numClasses} %loop vertical starting on top
{
	% Add class name on the left
	\node [anchor=east] at (0.4,-\y) {\pgfmathparse{\classNames[\y-1]}\pgfmathresult}; 
	
	\foreach \x in {1,...,\numClasses}  %loop horizontal starting on left
	{
		%---- Start of automatic calculation of totSamples for the column ------------   
		\def\totSamples{0}
		\foreach \ll in {1,...,\numClasses}
		{
			\pgfmathparse{\myConfMat[\ll-1][\x-1]}   %fetch next element
			\xdef\totSamples{\totSamples+\pgfmathresult} %accumulate it with previous sum
			%must use \xdef fro global effect otherwise lost in foreach loop!
		}
		\pgfmathparse{\totSamples} \xdef\totSamples{\pgfmathresult}  % put the final sum in variable
		%---- End of automatic calculation of totSamples ----------------
		
		\begin{scope}[shift={(\x,-\y)}]
		\def\mVal{\myConfMat[\y-1][\x-1]} % The value at index y,x (-1 because of zero indexing)
		\pgfmathtruncatemacro{\r}{\mVal}   %
		\pgfmathtruncatemacro{\p}{round(\r/\totSamples*100)}
		\coordinate (C) at (0,0);
		\ifthenelse{\p<50}{\def\txtcol{black}}{\def\txtcol{white}} %decide text color for contrast
		\node[
		draw,                 %draw lines
		text=\txtcol,         %text color (automatic for better contrast)
		align=center,         %align text inside cells (also for wrapping)
		fill=black!\p,        %intensity of fill (can change base color)
		minimum size=\myScale*10mm,    %cell size to fit the scale and integer dimensions (in cm)
		inner sep=0,          %remove all inner gaps to save space in small scales
		] (C) {\r\\\p\%};     %text to put in cell (adapt at will)
		%Now if last vertical class add its label at the bottom
		\ifthenelse{\y=\numClasses}{
			\node [] at ($(C)-(0,0.75)$) % can use vertical or diagonal label as option
			{\pgfmathparse{\classNames[\x-1]}\pgfmathresult};}{}
		\end{scope}
	}
}
%Now add x and y labels on suitable coordinates
\coordinate (yaxis) at (-0.3,0.5-\numClasses/2);  %must adapt if class labels are wider!
\coordinate (xaxis) at (0.5+\numClasses/2, -\numClasses-1.25); %id. for non horizontal labels!
\node [vertical label] at (yaxis) {Classe Risultante};
\node []               at (xaxis) {Classe Target};
\end{tikzpicture}

È possibile notare degli ottimi risultati per tutti gli strumenti nel riconoscimento, in particolar modo per la cassa e il colpo di rullante, vista la natura dell'onda sonora da loro prodotta, un'impulso con una crescita molto rapida. Lo strumento che è riconosciuto peggio è invece il rullante spazzolato, anch'esso per la natura dell'onda sonora da lui prodotto, distinguibile meno facilmente da un possibile rumore per ampiezza e impulso con crescita molto minore.\section{Seconda fase}