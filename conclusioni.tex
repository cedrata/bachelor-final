\chapter*{Conclusioni}
In questa tesi l'obiettivo è stato fornire un primo approccio, come scritto nell'introduzione, all'utilizzo del ML per la costruzione di uno strumento che permettesse di effettuare studi sul groove musicale provando dunque a insegnare a un a macchina a riconoscere una nota da una non-nota, per poterle collocare temporalmente per poter comprendere dunque la correlazione che esiste tra i vari strumenti i musicisti si incastrino temporalmente tra loro. Dopo aver analizzato il problema e trovato un modo teorico per la risoluzione, si è passati alla pratica con l'esperimento descritto nel terzo capitolo agendo dunque in due fasi: la prima nella quale si prova a comprendere l'abilità di una macchina di portare a termine questo tipo di riconoscimento, e una seconda nella quale viene tentato un approccio più fino. I risultati che sono stati ottenuti nella prima parte dell'esperimento sono molto più che positivi, con dei valori di TPR e di ROC Area superiori addirittura al 90\% mostrando dunque che i classificatori ottenuti per i diversi strumenti si sono rivelati efficaci. La prima fase è stata dunque trattata in modo corretto, trattando il problema 