\chapter*{Conclusioni}
In questa tesi l'obiettivo è stato fornire un primo approccio, come scritto nell'introduzione, all'utilizzo del ML per la costruzione di uno strumento che permettesse di effettuare studi sul groove musicale provando dunque a insegnare a una macchina la differenza tra una nota da una non-nota, per poterle collocare temporalmente permettendo di svolgere un'analisi tra la correlazione che esiste tra i vari strumenti e come i musicisti si incastrino temporalmente tra loro creando una sensazione di inerzia nella musica. Dopo aver analizzato il problema e trovato un modo teorico per la risoluzione, si è passati alla pratica con l'esperimento descritto nel terzo capitolo agendo dunque in due fasi: la prima nella quale si prova a comprendere l'abilità di una macchina di portare a termine questo tipo di riconoscimento, e una seconda nella quale viene tentato un approccio più fino. I risultati che sono stati ottenuti nella prima parte dell'esperimento sono molto incoraggianti, vediamo infatti che si tratta di un problema di classificazione come intuito inizialmente dal momento che l'output deve essere un valore di tipo discreto ``si'' o ``no'' e i numeri ottenuti supportano quest'affermazione. In particolar modo notiamo dei valori di TPR e di ROC Area superiori addirittura al 90\% mostrando dunque un'elevata efficacia dei classificatori costruiti per i diversi strumenti.\\
Per quanto riguarda la seconda parte dell'esperimento risulta evidente che il problema non è stato trattato nel modo ideale, e differentemente da quanto ipotizzato inizialmente si tratta di un problema di regressione e non di uno di classificazione. Analizzando i risultati ottenuti si nota immediatamente che i valori ricavati sono decisamente più eterogenei rispetto alla prima fase dell'esperimento con dei valori di ROC Area che variano da circa il 0.5 fino al 0.9 circa, il che implica un algoritmo poco consistente che non permette di trovare classificatori con performance buone in modo consistente al variare dello strumento analizzato. In modo errato è stato deciso di assegnare delle classi con valori da 1 a 5 rappresentanti una percentuale di essere nota o meno, ma questo è chiaramente sconveniente, sarebbe possibile trattare il problema come un tipo di regressione, e probabilmente già questo potrebbe contribuire a un miglioramento del programma, assieme ad altre features e strumenti di analisi del suono più precisi che consentano di riconoscere la timbrica e altri aspetti legati al suono.\\
Ciò detto, come appunto ripetuto più volte, questa vuole essere la presentazione di un primo approccio che utilizza il ML in questo modo che si spera possa essere migliorato in futuro.

%in fine direi che si parla di regressione dal momento che abbiamo creato molteplici classi che indicano l'appartenenza a una percentuale di essere nota o meno. È stato però discretizzato il concetto di percentuale, COSA INUTILE.
% fase osserviamo che probabilmente l'approccio scelto non è adeguato al tipo di problema. % parla di come il problema è stato considerato come un tipo di oclassificazione ma probabilmente è più adeguato un approccio di tipo regressione dal momento che abbiamo delle "probabilità" di essere nota che noi abbiamo deciso di racchiudere dentro a delle classi 5, 4, 3, 2, 1. Abbamo dunque pensato di discrretizzare in più classi diverse. NON ADATTO APPROCCIO...