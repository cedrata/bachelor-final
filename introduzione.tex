\chapter*{Introduzione}
\label{cap:introduzione}
\addcontentsline{toc}{chapter}{Introduzione}
\lhead{\bfseries INTRODUZIONE}
\rhead{\thepage}

%%% Fancy header settings, queste impostazioni vanno fatte solo una volta all'inizio del primo capitolo!
\pagestyle{plain}
\fancyhf{}
\renewcommand{\headrulewidth}{2pt}
\fancyhead[EL]{\textbf{\textsf{\nouppercase\thepage}}}
\fancyhead[ER]{\textbf{\textsf{\nouppercase\leftmark}}}
\fancyhead[OR]{\textbf{\textsf{\nouppercase\thepage}}}
\fancyhead[OL]{\textbf{\textsf{\nouppercase {\rightmark}}}}
%%% end

%%% all'inizio di ogni capitolo, questa impostazione rimuove il numero di pagina, provare a commentare per vedere la differenza
\thispagestyle{empty}

Elementi essenziali nella musica sono importanti il tempo, la sincronia tra i diversi musicisti e come questi si interfacciano gli uni agli altri. Tra musicisti (principalmente in generi come il Jazz) si dice \emph{``avere groove''}, termine usato per definire un portamento ritmico che provoca nella musica un'empatia tra musicisti e ascoltatori\nocite{wiki:groove}.\\
Sembrerebbe che questa carica di emozioni possa trovare origine in un comportamento dei musicisti, i quali appaiono suonare non perfettamente a metronomo, ma gli stessi suonano introducendo delle micro-variazioni di tempo (di millisecondi) le singole note, originando del movimento all'interno della musica, questo fenomeno è chiamato \emph{microtiming}, e apparirebbe essere collegato al groove\nocite{8350302}.\\
Illustratomi questo problema dal mio tutore di tirocinio, il mio compito è stato quello di costruire un applicativo in grado di aiutare nell'analisi di questo problema. Dunque il risultato finale è un software in grado di collocare temporalmente le note suonate da uno strumento in una traccia audio dello stesso. Dopo un'analisi attenta, il percorso scelto per la risoluzione è stato quello dell'intelligenza artificiale, precisamente il ML (machine learning), tecnica molto usata in questo ambito \nocite{Wright06towardsmachine}per approcciarsi a questa analisi. Lo scopo era dunque insegnare a una macchina la differenza tra una nota e una non-nota all'interno delle tracce audio fornite.\\
Una volta costruita l'intelligenza del software usando il linguaggio di programmazione python, e appoggiandosi a \href{https://www.cs.waikato.ac.nz/ml/weka/}{weka}, software open source per l'apprendimento automatico, è risultato necessario aggiungere uno scheletro per facilitarne l'utilizzo a utenti meno esperti con linea di comando, dunque creare una GUI (Graphical User Interface), realizzata anch'essa in linguaggio Python e con un framework chiamato \href{https://www.qt.io/}{Qt}.\\

Oggi giorno esistono una quantità innumerabile di soluzioni e programmi che consentono di portare a termine i compiti più disparati, e proprio per questo il primo passo posto ancora prima di cominciare è stato quello di documentarsi a riguardo cercando se esistessero già dei SW che in grado di svolgere ciò che mi è stato chiesto, e una volta visto che nessuna delle soluzioni esistenti soddisfaceva il mio tutore è stato intrapreso il percorso di documentazione per la realizzazione del SW. Il problema è molto noto nell'ambito musicale ma sfortunatamente la letteratura, nonostante sia molta, per il lato informatico è molto poco sviluppata. Gli articoli però presenti hanno permesso di arrivare a comprendere come affrontare il problema, utilizzando appunto il ML, ciò nonostante le informazioni presenti non erano sufficienti, soprattutto per un utente inesperto in questa branca, non vi erano codici o spiegazioni dettagliate. Proprio per questo motivo questo è stato deciso di tentare un primo approccio da zero alla risoluzione di questo problema con il codice reso disponibile a chi volesse approfondire \href{https://github.com/cedrata/cedrata-detector}{alla mia pagina github} 