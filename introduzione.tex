\chapter*{Introduzione}
\label{cap:introduzione}
\addcontentsline{toc}{chapter}{Introduzione}
\lhead{\bfseries INTRODUZIONE}
\rhead{\thepage}

%%% Fancy header settings, queste impostazioni vanno fatte solo una volta all'inizio del primo capitolo!
\pagestyle{plain}
\fancyhf{}
\renewcommand{\headrulewidth}{2pt}
\fancyhead[EL]{\textbf{\textsf{\nouppercase\thepage}}}
\fancyhead[ER]{\textbf{\textsf{\nouppercase\leftmark}}}
\fancyhead[OR]{\textbf{\textsf{\nouppercase\thepage}}}
\fancyhead[OL]{\textbf{\textsf{\nouppercase {\rightmark}}}}
%%% end

%%% all'inizio di ogni capitolo, questa impostazione rimuove il numero di pagina, provare a commentare per vedere la differenza
\thispagestyle{empty}

Elementi essenziali nella musica sono il tempo, la sincronia tra i diversi musicisti e come questi si interfacciano gli uni agli altri. Tra musicisti (principalmente in generi come il Jazz) si dice \emph{``avere groove''}, termine usato per definire un portamento ritmico che provoca nella musica un'empatia tra musicisti e ascoltatori\nocite{wiki:groove}.\\
Questa carica di emozioni sembrerebbe essere data da un'imperfetta sincronia tra i musicisti, i quali parrebbero suonare non perfettamente a tempo. I musicisti stessi suonano le singole note introducendo delle micro-variazioni di tempo (millisecondi), originando in tal modo, del \emph{movimento} all'interno del brano musicale. Questo fenomeno è chiamato \emph{microtiming}, e questo, potrebbe essere alla base del groove\nocite{8350302}.\\
Partendo da questi presupposti, che il mio tutore di tirocinio aveva provveduto ad illustratimi, ho lavorato alla costruzione di un applicativo che potesse costituire uno strumento utile all'analisi di tale fenomeno. Il risultato finale è un software in grado di collocare temporalmente le note suonate da uno strumento in una traccia audio dello stesso. Dopo un'attenta analisi, il percorso scelto per la risoluzione è stato quello dell'intelligenza artificiale e precisamente il ML (machine learning), tecnica molto usata in questo ambito \nocite{Wright06towardsmachine}per approcciarsi a questa ambito. Lo scopo è stato dunque, quello di insegnare a una macchina la differenza tra una nota e una non-nota all'interno delle tracce audio fornite.\\
%Illustratomi questo problema dal mio tutore di tirocinio, il mio compito è stato quello di costruire un applicativo in grado di aiutare nell'analisi di questo problema. Dunque il risultato finale è un software in grado di collocare temporalmente le note suonate da uno strumento in una traccia audio dello stesso. Dopo un'analisi attenta, il percorso scelto per la risoluzione è stato quello dell'intelligenza artificiale, precisamente il ML (machine learning), tecnica molto usata in questo ambito \nocite{Wright06towardsmachine}per approcciarsi a questa analisi. Lo scopo era dunque insegnare a una macchina la differenza tra una nota e una non-nota all'interno delle tracce audio fornite.\\
Una volta costruita l'intelligenza del software usando il linguaggio di programmazione 
Python, e appoggiandosi a \href{https://www.cs.waikato.ac.nz/ml/weka/}{weka}, software open source per l'apprendimento automatico, è risultato necessario aggiungere uno scheletro che facilitasse l'utilizzo a utenti meno esperti nell'utilizzo della linea di comando, dunque creare una GUI (Graphical User Interface), realizzata anch'essa in linguaggio Python utilizzando un framework chiamato \href{https://www.qt.io/}{Qt}.\\

\nocite{AfroDeeplearningBeats} \nocite{QuantifyMicrotiming}
Oggi giorno esiste una quantità innumerabile di soluzioni e programmi che consentono di portare a termine i compiti più disparati e, proprio per questo, il primo passo posto ancora prima di cominciare è stato quello di documentarsi a riguardo, cercando se esistessero già dei SW in grado di svolgere ciò che mi è stato richiesto e, una volta verificato che nessuna delle soluzioni esistenti soddisfaceva il mio tutore ho intrapreso il percorso di documentazione per la realizzazione del SW. Il fenomeno è molto noto nell'ambito musicale ma, sfortunatamente, la letteratura in materia, nonostante sia vasta per il lato informatico è poco sviluppata. Gli articoli fino ad ora pubblicati hanno permesso di comprendere come affrontare il problema, utilizzando appunto il ML. Ciò nonostante le informazioni presenti non erano sufficienti, soprattutto per un utente inesperto in questa branca. Non vi erano codici o spiegazioni dettagliate e, proprio per questo motivo, è stato deciso di tentare un primo approccio alla risoluzione di questo problema con il codice reso disponibile a chi volesse approfondire \href{https://github.com/cedrata/cedrata-detector}{alla mia pagina github} partendo da zero  